\section*{Ejercicio 4}

\begin{enumerate}
    \item \textbf{Falso}. Sea $f = n^2$, $g = n^3$. Se verifica que $f = n^2 = O(n^3) = O(g)$ y también que $g = n^3 = \Omega(n^2) = \Omega(f)$. Pero no vale que $f = n^2 \neq \Theta(n^3) = \Theta(g)$ ya que no se puede acotar $f = n^2$ por arriba y por abajo $\pm$ constantes con $g = n^3$.
    \item \textbf{Verdadero}. Sea $f = n$ la complejidad del mejor caso del InsertionSort (cuando la entrada ya está ordenada y por lo tanto el ciclo interno no tiene que hacer nada y solo recorremos 1 vez el arreglo con el ciclo externo). Suponiendo ahora una entrada arbitraria, se verifica que el algoritmo tenga complejidad $\Omega(n)$, pues el mejor caso define la cota inferior de complejidad. Para una entrada arbitraria, si ésta resulta ser un caso promedio o el peor caso, el algoritmo puede resultar más ineficiente pero la cota inferior sigue siendo válida.
    \item \textbf{Falso}. Siguiendo con el ejemplo del InsertionSort, sea $f = n^2$ la complejidad del peor caso (cuando la entrada está ordenada al revés). Supongamos que la entrada es ahora el mejor caso, donde ya vimos que la complejidad es $O(n)$. Por definición, un algoritmo tiene complejidad $\Theta(f)$ si se puede acotar por arriba y por abajo con $f$, o dicho de otra forma, si el algoritmo es $\Omega(f) = O(f)$. Pero como hemos visto en el ejemplo, InsertionSort es $\Omega(n)$ en el mejor caso y $O(n^2)$ en el peor, por lo tanto el enunciado es falso.
    \item \textbf{Falso}. La complejidad del mejor caso de un algoritmo para un cierto problema es mayor \textbf{o igual} que cualquier límite inferior para el problema. Por ejemplo, cualquier algoritmo de ordenamiento basado en comparaciones tiene demostrado que la cota inferior es $\Omega(n log(n))$. Esto significa que la complejidad óptima que podemos lograr con este tipo de algoritmos es $\Omega(n log(n))$, no obstante puede haber (y hay) algoritmos del mismo tipo pero más ineficientes.
    \item \textbf{Falso}. Sean $S = [7, 3, 6, 1, 2, 4, 5], i = 4, j = 5$. Veamos que $i < j$ y que $S[i] = 2 < 4 = S[j]$. Si intercambiamos $S[i]$ por $S[j]$ se rompe el heap.
    \item \textbf{Falso}. Sean $S = [7, 3, 6, 1, 2, 4, 5], i = 2, j = 4$. Veamos que $i < j$ y que $S[i] = 6 > 2 = S[j]$. Si intercambiamos $S[i]$ por $S[j]$ se rompe el heap.
\end{enumerate}
