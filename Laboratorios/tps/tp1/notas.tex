\begin{itemize}
    \item Unas horas antes de la entrega del TP se aclaró lo siguiente: 'tanto agregar una casa o n, en una llamada a funcion, ambas tienen que ser O(1) en el SC'. Nuestro entendimiento del enunciado y lo que resolvimos en esta entrega es que agregar una casa tiene complejidad O(1), y agregar n casas tiene complejidad O(n) (hablando de las funciones propias del SimCity). A nuestro criterio, el enunciado admitía perfectamente este entendimiento. No tenemos tiempo para modificar nuestra entrega, pero describimos a continuación los cambios que haríamos si tuviésemos que garantizar esa complejidad. La estructura del SimCity ya no tendría 2 diccionarios independientes para guardar las posiciones de las casas y comercios. En cambio, tendría una única variable $construciones: lista(dicc(pos, construccion))$ en donde se agrega en O(1) todas las construcciones del turno cuando se ejecuta AvanzarTurno. De esta forma mantenemos la misma interfaz del módulo, y en las funciones Casas y Comercios, antes de realizar los algoritmos planteados, obtendríamos de la lista de construciones todas las casas o comercios construidos en todos los turnos que pasaron.
    \item En el cálculo de complejidades abusamos de notación y usamos el operador cardinal ($\#$) tanto para conjuntos como diccionarios. En el caso de un diccionario $d$ cualquiera, se debe interpretar la notación de esta forma: $\#d \equiv \#claves(d)$.
    \item La lógica que tomamos para resolver los conflictos es quedarnos con la construcción de mayor nivel, y ante un empate entre casa y comercio, nos quedamos con el comercio.
\end{itemize}
