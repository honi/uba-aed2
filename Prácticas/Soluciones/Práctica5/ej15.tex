\section{Ejercicio 15: tiene agujero}

\begin{algorithm}[H]
\caption{
    \textbf{TieneAgujero}(\textbf{in} A: arreglo(nat)) $\to$ \textbf{out} res: bool
}
\begin{algorithmic}[1]
    \State n $\gets$ tam(A) \Comment{$O(1)$}
    \State min $\gets$ BuscarMin(A) \Comment{$O(n)$}
    \State max $\gets$ BuscarMax(A) \Comment{$O(n)$}
    \If{max - min + 1 $>$ n}
        \State res $\gets$ true
    \Else
        \State C: arreglo(bool) $\gets$ CrearArreglo(n) \Comment{$O(n)$, elementos inicializados en false}
        \State r $\gets$ 0
        \For{i $\gets$ 1 \textbf{to} n} \Comment{$O(n)$}
            \State j $\gets$ A[i] - min + 1
            \If{C[j] = true}
                \State r $\gets$ r + 1
            \Else
                \State C[j] $\gets$ true
            \EndIf
        \EndFor
        \State res $\gets$ max - min + 1 + r $\neq$ n
    \EndIf
\end{algorithmic}
\Complexity{$O(n)$}
\end{algorithm}

Que no haya agujeros en esencia es que el arreglo tenga todos números consecutivos.

Sabiendo el mínimo y el máximo del arreglo podemos calcular la cantidad de números consecutivos sin repetidos: $max - min + 1$. Sea $r$ la cantidad de elementos repetidos (sin contar su aparición inicial), sabemos que hay agujeros si se cumple $max - min + 1 + r \neq n$.

Para contar los repetidos hacemos algo parecido a la parte de counting de un CountingSort, en donde recorremos una vez el arreglo de entrada y vamos contando cada vez que encontramos un número repetido.
