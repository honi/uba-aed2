\section{Ejercicio 14: múltiplos}

\begin{algorithm}[H]
\caption{
    \textbf{OrdenarMultiplos}(\textbf{in} A: arreglo(nat), \textbf{in} k: nat) $\to$ \textbf{out} res: arreglo(nat)
}
\begin{algorithmic}[1]
    \State MergeSort(A) \Comment{$O(n log(n))$}
    \State Ak: arreglo(arreglo(nat)) $\gets$ CrearArreglo(k) \Comment{$O(k)$}
    \For{i $\gets$ 1 \textbf{to} k} \Comment{$O(k)$}
        \State Ak[i] $\gets$ MultiplicarTodos(A, i) \Comment{$O(n)$}
    \EndFor
    \State res $\gets$ UnirOrdenados(Ak) \Comment{$O(nk log(k)) = O(nk log(n))$}
\end{algorithmic}
\Complexity{$O(nk log(n))$}
\end{algorithm}

\begin{algorithm}[H]
\caption{
    \textbf{MultiplicarTodos}(\textbf{in} A: arreglo(nat), \textbf{in} k: nat) $\to$ \textbf{out} res: arreglo(nat)
}
\begin{algorithmic}[1]
    \State n $\gets$ tam(A)
    \State res: arreglo(nat) $\gets$ CrearArreglo(n) \Comment{$O(n)$}
    \For{i $\gets$ 1 \textbf{to} n} \Comment{$O(n)$}
        \State res[i] $\gets$ A[i] * k
    \EndFor
\end{algorithmic}
\Complexity{$O(n)$}
\end{algorithm}

Primero ordenamos el arreglo de entrada A. Luego creamos $k$ arreglos en Ak, donde el arreglo en la posición i-ésima Ak[i] es el arreglo de entrada A con todos sus elementos multiplicados por $i$. A partir de este punto estamos en las mismas condiciones del ejercicio 4, tenemos un conjunto de arreglos ordenados, todos de tamaño $n$, y queremos unirlos de forma ordenada. Así que aprovechamos esa solución y simplemente usamos la función UnirOrdenados.
